% Options for packages loaded elsewhere
\PassOptionsToPackage{unicode}{hyperref}
\PassOptionsToPackage{hyphens}{url}
%
\documentclass[
  11pt,
]{article}
\usepackage{amsmath,amssymb}
\usepackage{lmodern}
\usepackage{iftex}
\ifPDFTeX
  \usepackage[T1]{fontenc}
  \usepackage[utf8]{inputenc}
  \usepackage{textcomp} % provide euro and other symbols
\else % if luatex or xetex
  \usepackage{unicode-math}
  \defaultfontfeatures{Scale=MatchLowercase}
  \defaultfontfeatures[\rmfamily]{Ligatures=TeX,Scale=1}
\fi
% Use upquote if available, for straight quotes in verbatim environments
\IfFileExists{upquote.sty}{\usepackage{upquote}}{}
\IfFileExists{microtype.sty}{% use microtype if available
  \usepackage[]{microtype}
  \UseMicrotypeSet[protrusion]{basicmath} % disable protrusion for tt fonts
}{}
\makeatletter
\@ifundefined{KOMAClassName}{% if non-KOMA class
  \IfFileExists{parskip.sty}{%
    \usepackage{parskip}
  }{% else
    \setlength{\parindent}{0pt}
    \setlength{\parskip}{6pt plus 2pt minus 1pt}}
}{% if KOMA class
  \KOMAoptions{parskip=half}}
\makeatother
\usepackage{xcolor}
\IfFileExists{xurl.sty}{\usepackage{xurl}}{} % add URL line breaks if available
\IfFileExists{bookmark.sty}{\usepackage{bookmark}}{\usepackage{hyperref}}
\hypersetup{
  pdftitle={Genetic architecthure of alternative reproductive tactics},
  hidelinks,
  pdfcreator={LaTeX via pandoc}}
\urlstyle{same} % disable monospaced font for URLs
\usepackage{longtable,booktabs,array}
\usepackage{calc} % for calculating minipage widths
% Correct order of tables after \paragraph or \subparagraph
\usepackage{etoolbox}
\makeatletter
\patchcmd\longtable{\par}{\if@noskipsec\mbox{}\fi\par}{}{}
\makeatother
% Allow footnotes in longtable head/foot
\IfFileExists{footnotehyper.sty}{\usepackage{footnotehyper}}{\usepackage{footnote}}
\makesavenoteenv{longtable}
\usepackage{graphicx}
\makeatletter
\def\maxwidth{\ifdim\Gin@nat@width>\linewidth\linewidth\else\Gin@nat@width\fi}
\def\maxheight{\ifdim\Gin@nat@height>\textheight\textheight\else\Gin@nat@height\fi}
\makeatother
% Scale images if necessary, so that they will not overflow the page
% margins by default, and it is still possible to overwrite the defaults
% using explicit options in \includegraphics[width, height, ...]{}
\setkeys{Gin}{width=\maxwidth,height=\maxheight,keepaspectratio}
% Set default figure placement to htbp
\makeatletter
\def\fps@figure{htbp}
\makeatother
\setlength{\emergencystretch}{3em} % prevent overfull lines
\providecommand{\tightlist}{%
  \setlength{\itemsep}{0pt}\setlength{\parskip}{0pt}}
\setcounter{secnumdepth}{-\maxdimen} % remove section numbering
\newlength{\cslhangindent}
\setlength{\cslhangindent}{1.5em}
\newlength{\csllabelwidth}
\setlength{\csllabelwidth}{3em}
\newlength{\cslentryspacingunit} % times entry-spacing
\setlength{\cslentryspacingunit}{\parskip}
\newenvironment{CSLReferences}[2] % #1 hanging-ident, #2 entry spacing
 {% don't indent paragraphs
  \setlength{\parindent}{0pt}
  % turn on hanging indent if param 1 is 1
  \ifodd #1
  \let\oldpar\par
  \def\par{\hangindent=\cslhangindent\oldpar}
  \fi
  % set entry spacing
  \setlength{\parskip}{#2\cslentryspacingunit}
 }%
 {}
\usepackage{calc}
\newcommand{\CSLBlock}[1]{#1\hfill\break}
\newcommand{\CSLLeftMargin}[1]{\parbox[t]{\csllabelwidth}{#1}}
\newcommand{\CSLRightInline}[1]{\parbox[t]{\linewidth - \csllabelwidth}{#1}\break}
\newcommand{\CSLIndent}[1]{\hspace{\cslhangindent}#1}
\usepackage{lipsum} \usepackage{float} \floatplacement{figure}{H}

\usepackage{booktabs}
\usepackage{longtable}
\usepackage{array}
\usepackage{multirow}
\usepackage{wrapfig}
\usepackage{float}
\usepackage{colortbl}
\usepackage{pdflscape}
\usepackage{tabu}
\usepackage{threeparttable}
\usepackage{threeparttablex}
\usepackage[normalem]{ulem}
\usepackage{makecell}
\usepackage{xcolor}
\ifLuaTeX
  \usepackage{selnolig}  % disable illegal ligatures
\fi

\title{Genetic architecthure of alternative reproductive tactics}
\author{true \and true}
\date{}

\begin{document}
\maketitle

\hypertarget{introduction}{%
\section{Introduction}\label{introduction}}

Both genetic and phenotypic variation are required for evolution to
occur, and as such the maintenance of variation in populations has long
been of interest to evolutionary biologists and conservation
geneticists. Evolution is often expected to remove variation from
populations, with specific patterns of selection being predicted to
maintain variation at a high frequency -- including overdominance,
sexually antagonistic selection
{[}1,\textbf{connallonResolutionSexualAntagonism2011?}{]}, and frequency dependent
selection {[}2{]}. Yet, substantial phenotypic
variation is observed, with ubiquitous examples such as the prevalence
of sexual dimorphism and the unexpectedly frequent existence of
alternative reproductive tactics, in which individuals of the same
species and sex display different morphs to maximize reproductive
fitness in unique ways {[}3{]}.

The genetic basis of polymorphic traits is also critical to
understanding the maintenance of variation. Evolution can be constrained
by factors including pleiotropy, genetic correlations, and physical
linkage. As such, the reduction or removal of these constraints can
facilitate sequence evolution by relaxing positive selection on genes
{[}4,\textbf{dapperRelaxedSelectionRapid2020?}{]}.
However, how genome-wide patterns of constraining factors and patterns of sex- or morph-specific selection remains under-explored in evolutionary theory, in particular as it relates to creating baseline predictions for interpreting empirical datasets.
Specifically, the majority of predictions relating selection to genetic
constraints arise from one- or two-locus models (EXAMPLES), which can be
difficult to extrapolate to genome-wide predictions. In the modern
genomic era, having predictions for large genome-wide datasets is of
critical importance to develop testable hypotheses (citations).

One genetic mechanism that is increasingly being considered as an important
mechanism in evolution are structural variants, such as inversion
polymorphisms that harbour so-called `supergenes'. These variants are
regions of the genome in which some individuals carry an inversion, in
which genes with direct phenotypic effects exist. The mismatched
directionality reduces recombination between the different variants,
facilitating the maintenance of both genetic and phenotypic variations.
These supergenes have been shown to underpin stable polymporhisms
including multiple social morphs in fire ants
{[}@yanEvolutionSupergeneThat2020{]}, colony sex ratio skew in ants
{[}@lagunas-roblesLinkedSupergenesUnderlie2021{]}, sex-specific migratory behaviours {[}@pearseSexdependentDominanceMaintains2019{]}, sperm length and swim speed {[}@kimSexlinkedSupergeneControls2017; @kniefSexchromosomeInversionCauses2017{]}, venom composition in rattlesnakes
{[}@zancolliWhenOnePhenotype2019{]}, and sexually dimorphic alternative
reproductive tactics {[}@thomasChromosomalPolymorphismLinked2008;
@kupperSupergeneDeterminesHighly2016;
@lamichhaneyStructuralGenomicChanges2016;
@hendrickxMasculinizingSupergene2022{]}.The potential for structural
variants to underpin sexually dimorphic polymorphisms is intriguing, as
most research to date has focused on the role of differential gene
expression in underlying sexual dimorphism {[}citations{]}. For example, in
canaries, sexual dichromatism appears to be mediated by differential
expression of a single gene involved in carotenoid breakdown
{[}@gazdaGeneticMechanismSexual2020{]}. Androgens are a likely mechanism
for modulating these gene expression differences, for example in
primates male-biased genes are more likely to be proximal to androgen
response elements {[}@andersonRelationshipSexualDimorphism{]}. Unlike the
canaries, though, examples where supergenes apparently underpin sexually
dimorphic phenotypes tend to involve multiple morphs of one sex, usually
males. Nearly all examples include a female-mimic morph, including ruffs
{[}@kupperSupergeneDeterminesHighly2016;
@lamichhaneyStructuralGenomicChanges2016{]}, dwarf spiders
{[}@hendrickxMasculinizingSupergene2022{]}, and white-throated sparrows
{[}@thomasChromosomalPolymorphismLinked2008{]}. In the ruffs, a third male
morph exists, with an intermediate phenotype between the dominant or
`classic' male morph and the female-mimic
{[}@kupperSupergeneDeterminesHighly2016;
@lamichhaneyStructuralGenomicChanges2016{]}. At least in some cases,
these structural variants are successful in coding for these variable
phenotypes because they include androgen receptors -- the white-throated
sparrow supergene contains the estrogen receptor gene which directly
influences the aggressive aspects of the dominant male phenotype
{[}@merrittSupergenelinkedEstrogenReceptor2020{]}.

These stable polymorphisms -- underpinned by supergenes -- are
often maintained in populations through partitioning of reproductive
success over time (e.g., the feminized dwarf spider morph mates earlier,
whereas the alternative morph is able to mate with previously-mated
females later in the season {[}@hendrickxMasculinizingSupergene2022{]}),
frequency dependent selection (examples?). In some cases, a heterozgyous
genotype is lethal (e.g., ruffs {[}@kupperSupergeneDeterminesHighly2016;
@lamichhaneyStructuralGenomicChanges2016{]}), but not in others (e.g.,
dwarf spiders {[}@hendrickxMasculinizingSupergene2022{]}).

Supergenes and other structural variants might resolve interlocus? conflict, but they might also exacerbate it.

What about conflict between morphs?

One way that researchers have observed a role for supergenes is through genomic analysis. Putative regions are usually first identified through estimating allele frequency differences between morphs, testing for associations with the traits of interest (e.g., {[}@lagunas-roblesLinkedSupergenesUnderlie2021{]}). Recent mathematical and simulation models have helped clarify the allelic variation that is expected to be maintained
under sex-specific selection pressures,
and importantly have suggested that sex-specific selection must be incredibly strong
{[}5{]}.
However, providing direct linkages between theoretical predictions of
evolutionary mechanisms, genotype-to-phenotype maps, and realistic
datasets are currently lacking. Here, we fill this gap using
individual-based simulation models of alternative reproductive tactics
in males and three types of genetic architectures for the traits of
interest. \textbf{Not sure if this is useful}

An open question is to what extent do supergenes facilitate the maintenance of multiple morphs, versus whether the evolution of multiple morphs might facilitate the evolution of supergenes.
This question is difficult to address empirically, and is likely to have consequences for the maintenance of genetic variation at loci contributing to traits. For example, if a supergene exists and a mutation or arises within it that impacts the morph traits, the genetic variation within the supergene itself could have eroded due to relaxation of selection and reduced recombination. As such, only the mutation would be likely to show significant genetic variation -- this could make a supergene with functional effects difficult to detect empirically. This is especially true in cases such as the white-throated sparrow, in which the gene with a major effect is an estrogen receptor {[}@merrittSupergenelinkedEstrogenReceptor2020{]}, which likely impacts the expression of many genes.
To tackle this problem, we have developed a simulation model incorporating explicit genetic architectures.
We first identify regions of the frequency-dependent parameter space that should result in the maintenance of multiple morphs in a single-locus context using a mathematical model, and subsequently compare the outcomes of multilocus genetic architectures to those baselines.
Our results suggest that morphs encoded by many genes are more robust to stochastic elements that impact the morph frequencies over time, and that the number of morphs that are predicted to persist changes when multilocus genotypes are involved.
When the morphs have arrived at equilibrial frequencies, selection on male traits is generally relaxed, and this corresponds with a decreased ability to reliably detect causative variants using typical genomic analysis approaches (i.e., genome-wide associations).

\hypertarget{methods}{%
\section{Methods}\label{methods}}

Our model focusses on a single population of diploid males and females
with non-overlapping generations. We model two independent traits,
courtship and parental care, both of which are expressed only in males,
but are also inherited by females. As such, males are one of four
morphs: courter/parents; courter/non-parents; non-courters/parents; or
non-courters/non-parents. The model is inspired by fish species like the
ocellated wrasse, \emph{Symphodus ocellatus}, and as such we will use
terminology appropriate to that species. Nonetheless, our model can be
generalized to any species in which one sex (but not the other) has two
traits that impact mating success and the survival of their offspring.

Each generation has three distinct phases: (1) mate choice and
fertilization; (2) parental care; and (3) viability selection. In the
mate choice phase of the model, females always prefer to mate with males
displaying the courtship trait. If a female decides to mate with a male,
he receives all of her eggs to care for. The male is not guaranteed
paternity of his mates' entire brood, however, because non-courting
males are able to fertilize a proportion of the eggs. The chosen parent
does receive first-male priority. The consequences of this setup are
that the only mechanism by which non-courters can produce offspring is
through these sneak fertilizations. In the parental care phase of the
model, males with the parental care trait have 100\% nest survival while
males without the parental care trait have 0\% nest survival, independent
of the male's courtship trait. Consequently, non-parents will only be
sustained in the population by sneaking fertilizations in the parental
males' nests. Finally, viability selection is imposed on offspring as
they mature to adulthood, such that males carrying the courtship trait
and/or the parental care trait have a lower probability of survival than
females and males without either courtship or parenting. Males who are
both courters and parents have a multiplicative effect of viability
selection.

We implemented our model first as a mathematical model and then as an
individual-based simulation model. The purpose of the mathematical model
was to generate baseline predictions for when multiple morphs are
maintained.

\hypertarget{mathematical-model}{%
\subsection{Mathematical model}\label{mathematical-model}}

In the mathematical model, we model the proportion of the population
that belong to morphs at each stage of the lifecycle (Fig.
\ref{fig:mathModel}). In the first stage, we consider the proportion of
the population's nests that will be allocated to each morph. The maximum
number of nests that could be laid is proportional to the number of
males in the population (\(N_m\)), if each male has the same likelihood of
attracting a female to deposit eggs with him. However, in our model,
males are not equally likely to be chosen by females, and therefore the
maximum number of nests is scaled by the probability of females mating
with each type of male morph. In the case of our model, the total number
of nests laid in the population becomes \[
\frac{N_f}{N_mf_{CP}+N_m f_{CNp}}
\] because we have constrained females to only mate with courting males.

\begin{figure}[H]
\includegraphics[width=1\linewidth,]{frequencyDependence} \caption{Diagram of the mathematical model describing the inheritance of alternative male reproductive morphs given courtship and parenting traits.}\label{fig:mathModel}
\end{figure}

Likewise, the proportion of nests each type of male receives (\({p_n}_M\))
is determined by the number of morphs in the population (\(N_mf_M\) for
the \(M\) morph), the probability of a female choosing that morph
(\({w_s}_M\)), and the ratio of the number of nests to the number of
females (\(n/N_f\)).

Within a given nest, the offspring could either be fertilized by the
nest-holding male or non-courting males `sneaking' fertilizations.
Therefore, each morph has a proportion of eggs fertilized within their
own nests (\({p_{fn}}_M\)) and a proportion of eggs fertilized in other
males' nests (\({p_{fs}}_M\)). These proportions were determined by the
sperm competition coefficient, \(c\), which penalized non-courting males,
and the relative reproductive investment for that morph (\(r_M\)).
Analogously, the proportion of offspring in a nest surviving due to
parental care investment can be split into the proportion surviving in a
male's own nest (\({p_{sn}}_M\)) or in another male's nest (\({p_{ss}}_M\)),
which is simply the proportions from the fertilization step weighted by
the probability of the nest surviving if the morph is a parent (\(w_n=1\))
or of the morph is a non-parent (\(w_n=0\)). These proportions
(\({p_{sn}}_M\) and \({p_{ss}}_M\)) are summed to determine the the overall
proportion of embryos in the population that survive the parenting phase
in each morph's nests (\({p_s}_M\)).

The final stage of the lifecycle is for offspring surviving in the nest
to undergo viability selection by multiplying the viability selection
coefficient (\({w_v}_M\)) by the surviving proportion (\({p_s}_M\)) and the
original frequency of the morph in the population. Finally, the
frequency of each morph in the next generation was calculated by
dividing the surviving proportion (\({p_v}_M\) by the sum of all
proportions.

To evaluate the outcomes of this mathematical model, we iterated forward
in time for 100 generations from 1659 combinations of initial morph
frequencies and repeated those iterations across 5 values for the sperm
competition coefficient (\(c\) = \{0..1\}) and 21 values for the relative
reproductive coefficient (\(r_M\) = \{0\ldots2\}). The iterations were
implemented in R v. 4.1.1 {[}\textbf{rcoreteam2021?}{]} using some features of tidyR
{[}\textbf{wickham2021?}{]}. To facilitate visualisation of the results, we
calculated the Shannon's diversity index for the morphs remaining in the
100th generation using the R package vegan {[}\textbf{oksanen2015?}{]}. We then
visualised the diversity of the population using contour plots produced
by plotly {[}\textbf{sievert2020?}{]} in a shiny app that relies on R packages shiny
{[}\textbf{chang2021?}{]}, shinydashboard {[}\textbf{chang2018?}{]}, rsconnect {[}\textbf{mcpherson2021?}{]},
dplyr {[}\textbf{wickham2021?}{]}, and rmarkdown {[}\textbf{allaire2021?},\textbf{xie2020?},\textbf{xie2018?}{]}.
The interactive app is available at
\url{https://spflanagan.shinyapps.io/morph_predictions/}.

Note that this model does not incorporate recombination or inheritance
via females. Nonetheless, this model provides a baseline predictions for
parameter values likely to facilitate morph diversity within a
population. These limitations are addressed in the individual-based
simulation model.

\hypertarget{individual-based-simulation-model}{%
\subsection{Individual-based simulation model}\label{individual-based-simulation-model}}

To evaluate the role of genetic architecture in the maintenance of
alternative reproductive tactics, we wrote an individual-based
simulation model in C++. The simulation model follows the same lifecycle
as described above, but rather than tracking proportions of the
population belonging to particular morphs, we tracked individuals within
the population. Each individual contained diploid genotypes, which
included both non-coding loci and quantitative trait loci (QTLs). We
included two types of QTLs: ones that impacted the individual's courting
trait and ones that impacted the individual's parenting trait. The QTLs
were each assigned a randomly-drawn effect on the trait, and the QTLs
additively determined the trait value for each trait (i.e., the effects
of all QTLs were summed to determine the trait value). In this model, we
assume no environmental effects impacted the trait value. These trait
values were then compared to a threshold to dichotomize each trait. The
threshold for each trait was the mean trait value during the
initialization of the model (i.e., at generation 0).

We implemented three different genetic architectures: a single gene per
trait, multiple QTLs randomly placed throughout chromosomes, and
supergenes. The single gene scenario is most analogous to the
mathematical model described above. When the traits have multiple QTLs
determining their trait values, each QTL is assigned a location on a
chromosome and the QTL's effect on the trait is associated with a single
nucleotide polymorphism (SNP). These SNPs undergo recombination and
mutation during mating. The supergene scenario forced the QTLs to
cluster into a small proportion of one of the chromosomes. These QTLs
still experience mutation but experienced reduced recombination.

When we ran the model, we initialized four replicate populations with
the identical starting conditions to evaluate the impact of
stochasticity on the model. These identical conditions included
randomly-generated values for the allelic effects of QTLs, exact number
of individuals of each sex and morph, and thresholds for determining
morph traits (i.e., the values at which an individual is deemed a
courter vs a non-courter and a parent vs a non-parent). For each set of
four populations, we ran 10,000 burn-in generations followed by 2,000
`experimental' generations. Doing so ensured that the population reached
a steady state in which we could examine any stable polymorphisms or
cyclical fluctuations in morph frequencies.

Within each of those generations, the population underwent four main
phases: (1) mate choice and fertilization; (2) nest survival; (3)
viability selection on juveniles; (4) maturation to adulthood. The main
difference between the individual-based simulation model and the
mathematical model were the introduction of stochasticity during mate
choice and the introduction of recombination and mutation during
reproduction. During the mate choice phase, females searched through a
set number of males and chose the first attractive male (i.e., the first
courter male she encountered). If a female did not find an acceptable
mate, she either did not mate (`no mating' or `nm' scenarios) or she
chose a male at random with from the population and allocated all of her
offspring to him, depending on the parameter settings. Females always
preferred courting males.

Once a female chose a male, fertilization of the eggs occurred, in which
the chosen male had first male priority (i.e., the chosen male allocated
all of his reproductive effort to his own nest) and up to 3 non-courting
males could `sneak' fertilizations, with their contributions discounted
by a factor of \(c\). After the males were assigned their paternity
shares, the nests survived if the chosen male was also a parent.
Subsequently, the total number of offspring in the population were
created based on the proportion of population reproductive success
achieved by each individual male and female (i.e., each generation, the
number of offspring created was equal to the carrying capacity).

When fertilization occurred, the parents' genotypes (both neutral
markers and QTLs) had the opportunity to recombine. Recombination
occurred independently on the chromosome inherited from the mother and
the chromosome inherited from the father. The number of recombination
events for a given chromosome was determined by drawing from a Poisson
distribution with the recombination rate as the mean, which we set to
0.2. For each recombination event, breakpoints were chosen by randomly
drawing from the total number of markers on a chromosome. These
breakpoints defined the segments of the chromosome that experienced
recombination, and each segment was randomly chosen to be assigned the
genotypes and effects from the parents' maternal or paternal chromosome.
When the genetic architecture was that of a supergene, if the
breakpoints fell within the region containing the aggregated QTLs for
the traits, the offspring was considered not viable.

After recombination occurred, if the progeny survived, the genotypes
underwent mutation, which occurred at a rate of \(mu\). For a given
individual, the number of mutations introduced was determined as
\(2\mu*n_SNPs*n_chrom\). The location of each mutation was randomly
selected and the allele at that location was randomly chosen from the
set of existing alleles. The mutation also impacted the phenotypic
effects of QTLs by altering the original effect by a value drawn from a
normal distribution with a mean of 0 and a standard deviation of
\(sigma_\mu\). Finally, offspring were also randomly assigned a sex.

The next stage in the lifecycle was viability selection, which impacted
only male offspring. For each male offspring, a survival probability was
calculated as \(e^{\frac{-morph2}{2\omega_v}}\). Viability selection acted
multiplicatively on courter-parents (their survival probability was
\(e^{\frac{-morph2}{2\omega_v}}*e^{\frac{-morph2}{2\omega_v}}\)). A random
value between 0 and 1 was drawn, and if the number was larger than the
survival probability for an individual, that individual did not survive
to adulthood. If, at this point, the number of surviving offspring was
larger than the carrying capacity, offspring were randomly selected to
become adults. When offspring matured to adulthood, the previous
generation's adults were replaced, and the lifecycle began again.

Each set of genetic architectures and population parameters was also was
run with four types of mating systems, all of which required females to
lay their eggs with only a single male: males accept eggs from a single
female, and females deposit their eggs with a randomly-selected unmated
male in the population should they fail to find a preferred mate
(`monogamy'); males accept eggs from multiple females, and females
deposit their eggs with a randomly-selected unmated male in the
population should they fail to find a preferred mate (`polygyny'); males
accept eggs from a single female, and females who fail to find a
preferred mate do not produce offspring (`monogamy no mating'); males
accept eggs from multiple females, and females who fail to find a
preferred mate do not produce offspring (`polygyny no mating').

\hypertarget{model-analysis}{%
\subsubsection{Model analysis}\label{model-analysis}}

The outcomes of simulation model were first assessed by visualising the proportions of each morph that persisted to the final generation of the model.
The first priority was to compare the outcomes of the simulation model with a single locus for each trait to the results from the mathematical model.
Next, the four mating systems were compared in the single-locus model.
From this comparison, we were then able to narrow the parameter space for the comparisons of the two different multilocus genetic architectures.
The many different formats for these two genetic architectures were then compared in terms of the morph composition in the final generation, and the amount of variation among replicate runs.

To identify how selection acts on the two male traits, selection gradients on the courter trait and parent trait (both the binary variable of courter/non-courter and parent/non-parent) were estimated as the slope of a linear regression of lifetime reproductive success on each trait values. The patterns of selection were compared across genetic architectures and parameter settings.

\hypertarget{analysing-multilocus-genotypes}{%
\subsubsection{Analysing multilocus genotypes}\label{analysing-multilocus-genotypes}}

To provide a link between our theoretical models and the realities of
empirical data, we analyzed the genetic diversity of the populations at
the final generation of the individual based simulations. First we used
vcfR to estimate observed heterozygosity for all markers in the genome
and separately for the QTLs. Allele frequencies between morphs were also
compared using the Gst estimator in vcfR. Additionally, an association
between male phenotypes and genetic markers was tested for using
GWASpoly . For both the Gst estimates and the GWAS p-values, any major
peaks identified were scanned to identify if they were within a 50
basepair window of one of the true QTLs.
Finally, we used vcftools {[}@danecekVariantCallFormat2011{]} to test for signals of balancing selection at the QTL loci compared to non-QTL loci.

To investigate whether linkage disequilibrium persisted over time within our various genetic architectures, we used vcftools {[}@danecekVariantCallFormat2011{]} to estimate linkage disequilibrium between all loci. We then estimated summary statistics for each locus and compared linkage disequilibrium between QTLs and compared it to linkage disequilibrium among non-QTLs. Because we had far fewer QTLs than non-QTL loci in all of the iterations, we also used re-sampling of the non-QTLs to estimate the average linkage disequilibrium, using 999 resampled estimates with the actual number of QTLs in each replicate.

\hypertarget{results}{%
\section{Results}\label{results}}

\hypertarget{frequency-dependent-population-dynamics-model}{%
\subsection{Frequency-dependent population dynamics model}\label{frequency-dependent-population-dynamics-model}}

The initial frequency of each morph impacted whether multiple morphs
were maintained. In no cases was the courter-nonparent morph maintained.
This result makes intuitive sense, because if a female chooses to nest
with a courter-nonparent, the nest will always die (given our parameter
settings).

The parameters controlling sperm competition (\(c\)), relative
reproductive allocation (\(r\)), and the number of sneakers allowed also
impacted the diversity of the population after both 100 generations and
1000 generations. Some initial starting conditions took more than 100
generations to reach a consistent (i.e., equilibrial) state, but the
contributions of these three parameters remained qualitatively
consistent across 100 and 1000 generations. Reproductive allocations
corresponding to a range of approximately 0.5 to 1 (meaning that
courters invest half as much to just as much as non-courters in
producing sperm) resulted in the maximum amount of diversity, as long as
sperm competition (\(c\)) was above 0.2 (meaning that, if a non-courting
male produced 8 sperm and courting males produce 4, the non-courting
male would fertilize 0.2222222 or more of the
offspring). This importance of the sperm competition coefficient was
influenced by the number of sneakers allowed, with larger contributions
from an individual non-courting male (e.g., larger \(c\)) being required
with fewer sneakers (Fig. \ref{fig:mathParams}).

\begin{figure}[H]
\includegraphics{../figs/mathParams-1} \caption{Contour plots showing the diversity of the populations given equal initial starting frequencies of all four morphs, the relative reproductive investments (x-axis), sperm competition coefficients (y-axis), and number of sneakers allowed per nest (1 through 5, plots going from top to bottom).}\label{fig:mathParams}
\end{figure}

The regions of the parameter space the have non-zero diversity values
correspond to scenarios where three of the four morphs were retained
(courter-parents, non-courter-parents, and non-courter-non-parents). The
dominant morph differed with various parameter combinations. Assuming
equal starting frequencies, after 10000 generations, 18.8235294\% of
parameter combinations that retained polymorphism resulted in the
courter-parent morph comprising \(\le\) 50\% of the population. The
remaining 81.1764706\% of parameter combinations resulting in
polymorphism had more equal representation of all three morphs, with the
average proportion of the population being
30.6002608 \% (\(pm\)
1.2914667 \%) courter-parents,
34.7824745 \% (\(pm\)
0.7462257 \%) non-courter-parents, and
34.6172647 \% (\(pm\)
0.6771888 \%) non-courter-non-parents.

\hypertarget{individual-based-simulation-model-1}{%
\subsection{Individual-based simulation model}\label{individual-based-simulation-model-1}}

The results from the mathematical model were used to select parameter
settings for the individual-based simulation model, with one set
expected to produce low-diversity populations (i.e., those where only
one morph were retained) and another set expected to produce
high-diversity populations (i.e., where 2 or more morphs were retained).
The chosen parameter settings are shown in Table
\ref{tab:SimParamsTable}.

\begin{table}[H]

\caption{\label{tab:SimParamsTable}Table of parameters selected to create a high-diversity population. All morphs started at equal frequencies.}
\centering
\begin{tabular}[t]{lrr}
\toprule
Parameter & Low Diversity Value & High Diversity Value\\
\midrule
r (in shiny app) & 2.0 & 0.75\\
r\_CP & 8.0 & 6.00\\
r\_CN & 8.0 & 6.00\\
r\_NP & 4.0 & 8.00\\
r\_NN & 4.0 & 8.00\\
\addlinespace
c & 0.5 & 0.70\\
number of sneakers & 2.0 & 2.00\\
\bottomrule
\end{tabular}
\end{table}

\hypertarget{impacts-of-mating-system-on-the-model}{%
\subsubsection{Impacts of mating system on the model}\label{impacts-of-mating-system-on-the-model}}

As expected, in the low diversity parameter space, the majority of
simulations resulted in fixation or near-fixation of the courter-parent
morph (Fig. \ref{fig:MatingTypesFig}). A more surprising result was the
inconsistency of results from the mating types within the the high
diversity parameter space. Polygyny resulted in the fixation of the
courter/parent morph regardless of whether females mated at random if a
suitable mate was found, and monogamy resulted in the fixation of the
non-courter/parent morph in every case. Only monogamy with no mating
allowed resulted in polymorphism, with the courter/parent morph
coexisting with either the non-courter/parent or the
non-courter/non-parent morph (Fig. \ref{fig:MatingTypesFig}). These
results can be explained by the nature of density-dependent survival in
the individual-based model that are not consistent with the mathematical
model; in the simulation model, offspring are created in proportion to
the surviving offspring after mating has occurred, increasing the males'
reproductive success above their specified reproductive investments.
This allows males to escape the costs and increase their reproductive
success beyond what is possible in the mathematical model. Random mating
if non-preferred males are unable to be found also enables
non-courter/parents to have increased reproductive success compared the
the mathematical model predictions, which is why they are fixed in the
monogamy case. If stricter costs on courting males are imposed,
specifying \(r=0.25\), meaning that \(r_{courter}=2\) and
\(r_{non-courter}=8\), the scenarios without random mating do result in
the maintenance of 2 or 3 morphs in most cases for both monogamy and
polygyny (Fig. \ref{fig:MatingTypesFig}), as expected.

\begin{figure}[H]
\includegraphics{../figs/MatingTypesFig-1} \caption{The frequencies of morphs in the final generation of simulation runs a single gene underlying each trait, with parameters expected to yield low diversity (top row), high diversity (centre row), and high diversity with stronger costs on courter reproductive success (bottom row). The frequency of the courter/parent morph in the final generation is printed on top of each bar. In 'monogamy' cases, the males were constrained to only accept eggs from one female into their nests, whereas under 'polygyny' males could mate with multiple females. In both cases females mated once. Parameter settings lablled with 'nm' were runs where females did not mate if they could not find a suitable mate (i.e., if no courters were available). Within each parameter set, the numbers below the bars identify sets of simulation runs that were initiated with identical starting conditions.}\label{fig:MatingTypesFig}
\end{figure}

One striking result is that stochasticity appears to play a major role
in which morphs co-exist with the courter-parent morph, even when
initial starting conditions are identical (Fig.
\ref{fig:MatingTypesFig}). Across independent iterations of the model,
variation in the initial starting frequencies of each morph (which
centre around 0.25 for each morph, but with some variation) can explain
some of the differences in outcomes. Variation in outcomes within
replicate runs can be attributed to the many sources of random chance
incorporated in the model: stochasticity associated with finite
population sizes (i.e., genetic drift), the randomness of recombination,
and the finite mate search process (in which females might not find a
suitable mate, even if one exists in the population) all contribute to
the within-replicate variation observed. Furthermore, these sources of
variation differentiate the results of the simulation model from those
of the mathematical model.

\hypertarget{selection-on-courter-and-parent-traits}{%
\subsubsection{Selection on courter and parent traits}\label{selection-on-courter-and-parent-traits}}

Selection consistently favoured parent traits, whereas courter traits could experience negative selection in high diversity parameter scenarios (Fig. \ref{fig:selectionOverTime}). Selection was strong in the first generation, but by generation 12000 selection was substantially relaxed on one or the other trait, in most cases.

\hypertarget{the-role-of-polygenic-inheritance-of-traits}{%
\subsubsection{The role of polygenic inheritance of traits}\label{the-role-of-polygenic-inheritance-of-traits}}

We incorporated polygenic inheritance of the male traits in two ways,
both of which reduced the impacts of stochasticity on the final morph
frequencies. In the low diversity parameter space, the outcomes remained
consistent with the single-locus model (and the expectations from the
mathematical model): courter/parents dominated the populations, with
complete fixation of the morph in the majority of runs, and any
variation existing in the form of 1-2 individuals (courter/parent
frequencies were \(\ge\) 99\% in all cases). These results were consistent
across all genetic architectures, including QTLs distributed across 2,
4, and 8 chromosomes and QTLs clustered in supergenes spanning from 5\%
to 50\% of a single chromosome. In the high diversity parameter space,
polygenic loci distributed throughout the genome resulted in the
maintenance of the courter/parent trait at a frequency of
50.7\%
(\(\pm\)
1.59\%),
with the non-courter/parent being the dominant morph. These proportions
are remarkably consistent across genetic architectures. Most simulations
with supergenes resulted in very similar final morph frequencies
(49.21\%
\(\pm\)
0.88\%),
and only occasionally resulted in either three morphs being retained or
the non-courter/non-parent morph replacing the non-courter parent. The
size of the supergene did not substantially impact the final morph
frequencies, nor did the total number of QTLs underlying each locus.

An important predictor of final morph frequencies was the direction of selection on the courter traits in the initial generation (Fig. \ref{fig:selFreq}).
Selection gradients were strongly positive on the parent trait in generation zero, whereas selection gradients were negative on a subset of the high diversity parameter runs (Fig. \ref{fig:selectionGradients}).
By the time that 12000 generations had passed, and the morphs had reached their equilibrial frequencies, selection was generally relaxed on the parental trait.
Selection on the courter trait was also relaxed in many cases (some points overlap 0).
A number of the low diversity replicates resulted in highly correlated selection on courters and parents, with a number of points falling on the positive 1:1 line (Fig. \ref{fig:selectionGradients}).

\begin{figure}[H]
\includegraphics{../figs/selFreq-1} \caption{The direction of selection on the courter trait in generation zero predicts whether diversity of morphs will be maintained.}\label{fig:selFreq}
\end{figure}

\begin{figure}[H]
\includegraphics{../figs/selectionGradients-1} \caption{Selection gradients on parent and courter traits differ across the scenarios of our simulation model. In the first generation, selection always favours the parent trait, but selection gradients on the courter trait can be negative under parameter settings that favoured high diversity. By the end of the simulations (generation 12000), selection has been relaxed, so that selection on the parent trait is often near zero, as is selection on the courter trait. In some cases, the selection gradients on courter and parent triats are correlated at this final generation, especially in low diversity scenarios with QTLs as the genetic architecture. Solid grey lines show the x- and y-axes, with dotted grey lines showing the one-to-one correlation lines.}\label{fig:selectionGradients}
\end{figure}

Balancing selection and/or maintenance of genetic variation? - genetic
variation at QTLs vs non QTLs vs detected qtls - Tajima's D?

Does linkage disequilibrium evolve?

\hypertarget{will-gwas-detect-the-actual-qtls}{%
\subsubsection{Will GWAS detect the actual QTLs?}\label{will-gwas-detect-the-actual-qtls}}

As expected, the detection of true QTLs in a GWAS was more likely when
fewer QTLs underpin the trait, especially if those QTLs are aggregated
on fewer chromosomes (Fig. \ref{fig:GWASoutcomes}). Somewhat
surprisingly, the QTLs for parent traits were more reliably detected by
GWAS than the courter QTLs. In the majority of simulations, the final
population only contained parental males, but both courters and
non-courters persisted in the population (Fig.
\ref{fig:MatingTypesFig}). The finding the parent QTLs are more likely
to be detected is likely because that trait experienced stronger
selection on it (i.e., being a non-parent was more costly than being a
non-courter).

We expected the supergene scenarios to have a higher detection rate, but
this was not the case. The primary difference was that the variance in
false positive rates was increased with supergenes, but the true
positive rates were unaffected by the localization of QTLs.

\begin{figure}[H]
\includegraphics{../figs/GWASoutcomes-1} \caption{Genome-wide association studies of the simulated genotypes reveals a relatively low detection rate of QTLs and a generally high false positive rate, both influenced by the genetic architecture of the traits. The top row show the results for genome-wide QTLs and the bottom row shows the results for supergenes.}\label{fig:GWASoutcomes}
\end{figure}

\begin{figure}[H]
\includegraphics{../figs/freqsDetection-1} \caption{The proportion of non-dominant morphs in the population did not impact the ability to detect QTLs through genome-wide associations. The top row shows multilocus QTL genetic architectures and the ability to detect traits considering either the parent/non-courters proportion or the non-courters/non-parent proportions. The bottom row shows the same results for supergenes.}\label{fig:freqsDetection}
\end{figure}

\hypertarget{discussion}{%
\section{Discussion}\label{discussion}}

Summary of findings - genetic architecture matters - maybe counter
intuitively it reduces stochasticity, likely through genetic
correlations and/or linkage disequilibrium? - GWAS is only going to pick
up \(\le\) 50\% of true QTLs, and will provide spurious results - a mention
of sexual conflict?

Why might cause the deviation from the single-locus expectation? - LD,
genetic correlations - compare to previous models and speculate on how
it might impact their results - discuss real-world examples

Detection of QTLs: what's up?
extreme LD within supergenes makes associations difficult to identify (e.g., Kim et al zebrafinch paper)
this is consistent with previous models
like my SCA paper - what does this mean for our ability to study real
world pops?

Maintenance of variation (esp.~morphs) and genetic architectures -
genetic architecture is critically important - this doesn't include
other elements of sex-specific or morph-specific architectures such as
sex-specific dominance of sex-biased gene expression

\hypertarget{acknowledgments}{%
\section{Acknowledgments}\label{acknowledgments}}

We acknowledge Ngāi Tūāhuriri, upon whose lands the analyses and writing
was conducted. This work was conducted in part while SPF was a
postdoctoral fellowship at the National Institute for Mathematical and
Biological Synthesis, sponsored by the National Science Foundation
through NSF Award \#DBI-1300426, with additional support from The
University of Tennessee, Knoxville. SPF was partially funded by the
Marsden fund grant number UOC1904 while working on this research.

\hypertarget{data-accessibility}{%
\section{Data Accessibility}\label{data-accessibility}}

All code has been archived on Zenodo (DOI: ) and is maintained on github
(). Simulation results are archived on dryad ().

\hypertarget{author-contributions}{%
\section{Author Contributions}\label{author-contributions}}

\hypertarget{references}{%
\section*{References}\label{references}}
\addcontentsline{toc}{section}{References}

\hypertarget{refs}{}
\begin{CSLReferences}{0}{0}
\leavevmode\vadjust pre{\hypertarget{ref-connallonEvolutionaryConsequencesSexSpecific2018}{}}%
\CSLLeftMargin{1. }
\CSLRightInline{Connallon T, Sharma S, Olito C. 2018 Evolutionary {Consequences} of {Sex-Specific Selection} in {Variable Environments}: {Four Simple Models Reveal Diverse Evolutionary Outcomes}. \emph{The American Naturalist} \textbf{193}, 93--105. (doi:\href{https://doi.org/10.1086/700720}{10.1086/700720})}

\leavevmode\vadjust pre{\hypertarget{ref-alonzoMateChoiceGames2001}{}}%
\CSLLeftMargin{2. }
\CSLRightInline{Alonzo SH, Sinervo B. 2001 Mate choice games, context-dependent good genes, and genetic cycles in the side-blotched lizard, {\emph{Uta}}{ \emph{Stansburiana}}. \emph{Behavioral Ecology and Sociobiology} \textbf{49}, 176--186. (doi:\href{https://doi.org/10.1007/s002650000265}{10.1007/s002650000265})}

\leavevmode\vadjust pre{\hypertarget{ref-oliveiraAlternativeReproductiveTactics2008}{}}%
\CSLLeftMargin{3. }
\CSLRightInline{Oliveira RF, Taborsky M, Brockmann HJ. 2008 \emph{Alternative {Reproductive Tactics}: {An Integrative Approach}}. {Cambridge, United Kingdom}: {Cambridge University Press}. }

\leavevmode\vadjust pre{\hypertarget{ref-harrisonSexualSelectionDrives2015}{}}%
\CSLLeftMargin{4. }
\CSLRightInline{Harrison PW, Wright AE, Zimmer F, Dean R, Montgomery SH, Pointer MA, Mank JE. 2015 Sexual selection drives evolution and rapid turnover of male gene expression. \emph{Proceedings of the National Academy of Sciences} \textbf{112}, 4393.}

\leavevmode\vadjust pre{\hypertarget{ref-kasimatisLimitsGenomicDivergence2019a}{}}%
\CSLLeftMargin{5. }
\CSLRightInline{Kasimatis KR, Ralph PL, Phillips PC. 2019 Limits to {Genomic Divergence Under Sexually Antagonistic Selection}. \emph{G3\&amp;\#58; Genes\textbar Genomes\textbar Genetics} \textbf{9}, 3813--3824. (doi:\href{https://doi.org/10.1534/g3.119.400711}{10.1534/g3.119.400711})}

\end{CSLReferences}

\end{document}
