% Options for packages loaded elsewhere
\PassOptionsToPackage{unicode}{hyperref}
\PassOptionsToPackage{hyphens}{url}
%
\documentclass[
]{article}
\usepackage{amsmath,amssymb}
\usepackage{lmodern}
\usepackage{iftex}
\ifPDFTeX
  \usepackage[T1]{fontenc}
  \usepackage[utf8]{inputenc}
  \usepackage{textcomp} % provide euro and other symbols
\else % if luatex or xetex
  \usepackage{unicode-math}
  \defaultfontfeatures{Scale=MatchLowercase}
  \defaultfontfeatures[\rmfamily]{Ligatures=TeX,Scale=1}
\fi
% Use upquote if available, for straight quotes in verbatim environments
\IfFileExists{upquote.sty}{\usepackage{upquote}}{}
\IfFileExists{microtype.sty}{% use microtype if available
  \usepackage[]{microtype}
  \UseMicrotypeSet[protrusion]{basicmath} % disable protrusion for tt fonts
}{}
\makeatletter
\@ifundefined{KOMAClassName}{% if non-KOMA class
  \IfFileExists{parskip.sty}{%
    \usepackage{parskip}
  }{% else
    \setlength{\parindent}{0pt}
    \setlength{\parskip}{6pt plus 2pt minus 1pt}}
}{% if KOMA class
  \KOMAoptions{parskip=half}}
\makeatother
\usepackage{xcolor}
\usepackage[margin=1in]{geometry}
\usepackage{graphicx}
\makeatletter
\def\maxwidth{\ifdim\Gin@nat@width>\linewidth\linewidth\else\Gin@nat@width\fi}
\def\maxheight{\ifdim\Gin@nat@height>\textheight\textheight\else\Gin@nat@height\fi}
\makeatother
% Scale images if necessary, so that they will not overflow the page
% margins by default, and it is still possible to overwrite the defaults
% using explicit options in \includegraphics[width, height, ...]{}
\setkeys{Gin}{width=\maxwidth,height=\maxheight,keepaspectratio}
% Set default figure placement to htbp
\makeatletter
\def\fps@figure{htbp}
\makeatother
\setlength{\emergencystretch}{3em} % prevent overfull lines
\providecommand{\tightlist}{%
  \setlength{\itemsep}{0pt}\setlength{\parskip}{0pt}}
\setcounter{secnumdepth}{-\maxdimen} % remove section numbering
\usepackage{booktabs}
\usepackage{longtable}
\usepackage{array}
\usepackage{multirow}
\usepackage{wrapfig}
\usepackage{float}
\usepackage{colortbl}
\usepackage{pdflscape}
\usepackage{tabu}
\usepackage{threeparttable}
\usepackage{threeparttablex}
\usepackage[normalem]{ulem}
\usepackage{makecell}
\usepackage{xcolor}
\ifLuaTeX
  \usepackage{selnolig}  % disable illegal ligatures
\fi
\IfFileExists{bookmark.sty}{\usepackage{bookmark}}{\usepackage{hyperref}}
\IfFileExists{xurl.sty}{\usepackage{xurl}}{} % add URL line breaks if available
\urlstyle{same} % disable monospaced font for URLs
\hypersetup{
  pdftitle={Ternary plots for ARTs results},
  hidelinks,
  pdfcreator={LaTeX via pandoc}}

\title{Ternary plots for ARTs results}
\author{}
\date{}

\begin{document}
\maketitle

\begin{verbatim}
## Warning: package 'knitr' was built under R version 4.2.2
\end{verbatim}

\begin{verbatim}
## Warning: package 'scales' was built under R version 4.2.2
\end{verbatim}

\begin{verbatim}
## Warning in !is.null(rmarkdown::metadata$output) && rmarkdown::metadata$output
## %in% : 'length(x) = 2 > 1' in coercion to 'logical(1)'
\end{verbatim}

\begin{verbatim}
## Warning: package 'plotly' was built under R version 4.2.2
\end{verbatim}

\begin{verbatim}
## Warning: package 'ggplot2' was built under R version 4.2.2
\end{verbatim}

\begin{verbatim}
## Warning: package 'dplyr' was built under R version 4.2.2
\end{verbatim}

\begin{verbatim}
## Warning: package 'vegan' was built under R version 4.2.2
\end{verbatim}

\begin{verbatim}
## Warning: package 'tidyr' was built under R version 4.2.2
\end{verbatim}

\begin{verbatim}
## Warning: package 'Ternary' was built under R version 4.2.2
\end{verbatim}

\hypertarget{baseline-model}{%
\subsection{Baseline model}\label{baseline-model}}

The baseline model results are saved in \texttt{morph\_results\_Ns.RDS},
I believe.

\includegraphics{../figs/baselineTernary-1}

Note: would be good to scale points based on number of times they land
at one outcome\ldots{}

\includegraphics{../figs/unnamed-chunk-1-1}

\hypertarget{single-locus-simulations}{%
\subsection{Single locus simulations}\label{single-locus-simulations}}

These simulations had three diversity settings but also four mating
systems, so I'll use a mix of colors and points to show the differences.

\includegraphics{../figs/singleLocusPlot-1}

\includegraphics{../figs/singleLocusSidBySide-1}

\includegraphics{../figs/monogamyPlot-1}

\hypertarget{polygenic-inheritance}{%
\subsection{Polygenic inheritance}\label{polygenic-inheritance}}

The main variables I want to show with the polygenic inheritance are the
genetic architecture (QTLs vs supergenes with diff props) and number of
chromosomes. I could also do number of QTLs -- that wasn't shown in the
main manuscript, as I showed representative images from the 8 QTLs
scenarios. How do I want to show this?

\begin{itemize}
\tightlist
\item
  In other figures, I've used circles and triangles for QTLs and
  supergenes, respectively
\item
  I could use colors to show num of chromosomes and do different
  triangles for each architecture
\item
  scale points based on the number of QTLs?
\end{itemize}

I'll start with just the QTL options.

\includegraphics{../figs/architectureTernary-1}

\hypertarget{a-combined-diagram}{%
\subsection{A combined diagram}\label{a-combined-diagram}}

I will create a combined diagram, with the top row showing the baseline
model, the single locus model, and QTLs, and the second row showing the
three supergene options. Maybe add a star to the baseline model to show
which parameter combination we selected to use for the simulations?

I'll only want to plot the monogamy\_nm ones from the single\_locus
runs.

\includegraphics{../figs/unnamed-chunk-6-1}

I'm not sure I like the fully combined one, as it hides some of the
important details (low vs high diversity settings, for example.) What if
instead I have two plots, one with the baseline and low/high/strict
settings?

\includegraphics{../figs/baselinesWithStrict-1}

\includegraphics{../figs/baselinesNoStrict-1}

\includegraphics{../figs/baselinesNoStrictPercentages-1}

And then to follow up on this one, we'll do a two-row genetic
architectures one

\includegraphics{../figs/architectureTernaryLowHigh-1}

\end{document}
