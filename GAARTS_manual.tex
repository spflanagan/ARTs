\documentclass[]{article}

%opening
\title{Genomic Architecture of Alternative Reproductive Tactics Simulation Model Handbook}
\author{Sarah P. Flanagan}

\begin{document}

\maketitle

\begin{abstract}

\end{abstract}

\section{Gene Network Model}
When using the gene network model, you specify both the total number of QTL (S) and the number of environmental QTL (E). The environmental QTL are a subset of the QTL so the number of non-environmentally responsive QTL is S-E.

There are several tracking things:

\begin{enumerate}
	\item int is an SxS matrix tracking the interactions among genes.
	\item Y is an SxS matrix with whether gene j regulates gene i.
	\item Z is a (S-E) vector of whether j contributes to trait i
	\item x is a (S-E) vector of the weights and contributions of gene j to trait i
\end{enumerate}

I have a separate one of each of these for each trait. Z and x could theoretically be combined into a matrix of vectors.

In each of these, the first E elements represent the environmentally responsive QTL.

Mutations affect either Y or Z. And the trait is determined only by the final values from the (S-E) QTLs.

QTLs have a randomly-chosen location on the chromosomes (which takes the place of a regular QTL - but they're not necessarily all lumped together).

\section{Supergenes}
Supergenes mean that all of the QTLs for the male trait(s) are grouped together on one chromosome. This is true regardless of whether it's a gene network model or not.

If a recombination event occurs within a supergene, the offspring does not survive. 

\section{Selection on Traits}

If all you do is specify that traits should exist, then courters/parents are favored during mating and/or reproductive success, but they have decreased survival as juveniles. 

For example,

./ARTs --courter

will result in courting males having a genetic architecture (but no other traits will have genetic architectures). Females will prefer courting males but courting males will have a lower probability of survival from juveniles to adults.

If you would like to have a courter trait but force random mating, you would run
./ARTs --courter --random-mating

The selection choices are as follows:
\begin{enumerate}
	\item Tradeoffs between sexual selection on courtship traits and viability (--courter)
	\item Tradeoffs between sexual selection on parental traits and viability (--parent)
	\item Tradeoffs between sexual selection on courtship traits, fecundity selection on parental traits, and viability selection on courtship and parental traits (--courter --parent)
	\item Frequency-dependent mate choice on either courtship or parental traits, where females prefer the less frequent morph (--courter --freq-dependent-preference OR --parent --freq-dependent-preference OR --courter --parent --freq-dependent-preference)
	\item Condition dependent preferences, where females choose which male morph she prefers based on her own fecundity. This choice imposes a female preference genetic architecture. (--courter --condition-dependent-preference OR --parent --condition-dependent-preference OR --courter --parent --condition-dependent-preference --correlated-pref)
	\item Frequency dependent viability selection against the courtship trait, where the most frequent morph has lower survival to adulthood (--courter --frequency-dependent-courter)
	\item Frequency dependent viability selection against the most frequent parental morph (--parent --frequency-dependent-parent)
	\item Condition dependent courtship traits, where courters are penalized during mating for courting and may end up switching to the non-courting morph if they have multiple opportunities to mate (--courter --condition-dependent-courter --polygyny)
	\item Condition dependent parental traits, where parents are penalized during mating for providing care and may end up switching to the non-parental morph if they have multiple opportunities to mate (--courter --condition-dependent-courter --polygyny)
\end{enumerate}

*if both courter and parent traits are called, mate choice always occurs on the courtship trait.

When --correlated-pref is called, that means that the QTLs underlying the male courtship trait are the same as the QTLs underlying the female preferences.

When calling the selection types, if you don't specify both the trait and the selection type then it will automatically call the trait as well. (e.g., --condition-dependent-courter is equivalent to --condition-dependent-courter --courter --polygyny). But note that calling only the trait only results in a tradeoff. 
 
\end{document}
