\documentclass[]{article}

%opening
\title{Genomic Architecture of Alternative Reproductive Tactics Simulation Model Handbook}
\author{Sarah P. Flanagan}

\begin{document}

\maketitle

\begin{abstract}

\end{abstract}

\section{Gene Network Model}
When using the gene network model, you specify both the total number of QTL (S) and the number of environmental QTL (E). The environmental QTL are a subset of the QTL so the number of non-environmentally responsive QTL is S-E.

There are several tracking things:

\begin{enumerate}
	\item int is an SxS matrix tracking the interactions among genes.
	\item Y is an SxS matrix with whether gene j regulates gene i.
	\item Z is a (S-E) vector of whether j contributes to trait i
	\item x is a (S-E) vector of the weights and contributions of gene j to trait i
\end{enumerate}

I have a separate one of each of these for each trait. Z and x could theoretically be combined into a matrix of vectors.

In each of these, the first E elements represent the environmentally responsive QTL.

Mutations affect either Y or Z. And the trait is determined only by the final values from the (S-E) QTLs.

\section{Supergenes}
Supergenes mean that all of the QTLs for the male trait(s) are grouped together on one chromosome. This is true regardless of whether it's a gene network model or not.



\end{document}
